\documentclass[a4paper,12pt]{article}

% В этом документе преамбула

%%% Работа с русским языком
\usepackage{cmap}					% поиск в PDF
\usepackage{mathtext} 				% русские буквы в формулах
\usepackage[T2A]{fontenc}			% кодировка
\usepackage[utf8]{inputenc}			% кодировка исходного текста
\usepackage[english,russian]{babel}	% локализация и переносы
\usepackage{indentfirst}
\frenchspacing

\renewcommand{\epsilon}{\ensuremath{\varepsilon}}
\renewcommand{\phi}{\ensuremath{\varphi}}
\renewcommand{\kappa}{\ensuremath{\varkappa}}
\renewcommand{\le}{\ensuremath{\leqslant}}
\renewcommand{\leq}{\ensuremath{\leqslant}}
\renewcommand{\ge}{\ensuremath{\geqslant}}
\renewcommand{\geq}{\ensuremath{\geqslant}}
\renewcommand{\emptyset}{\varnothing}
\newcommand*{\ooline}[1]{\overline{\overline{#1}}}
\newcommand*{\defe}{\overset{\underset{\mathrm{def}}{}}{=}}

%%% Дополнительная работа с математикой
\usepackage{amsmath,amsfonts,amssymb,amsthm,mathtools} % AMS
\usepackage{icomma} % "Умная" запятая: $0,2$ --- число, $0, 2$ --- перечисление

%% Номера формул
\mathtoolsset{showonlyrefs=true} % Показывать номера только у тех формул, на которые есть \eqref{} в тексте.
%\usepackage{leqno} % Нумереация формул слева

%% Свои команды
\DeclareMathOperator{\sgn}{\mathop{sgn}}

%% Перенос знаков в формулах (по Львовскому)
\newcommand*{\hm}[1]{#1\nobreak\discretionary{}
{\hbox{$\mathsurround=0pt #1$}}{}}

%%% Работа с картинками
\usepackage{graphicx}  % Для вставки рисунков
\graphicspath{{images/}{images2/}}  % папки с картинками
\setlength\fboxsep{3pt} % Отступ рамки \fbox{} от рисунка
\setlength\fboxrule{1pt} % Толщина линий рамки \fbox{}
\usepackage{wrapfig} % Обтекание рисунков текстом

%%% Работа с таблицами
\usepackage{array,tabularx,tabulary,booktabs} % Дополнительная работа с таблицами
\usepackage{longtable}  % Длинные таблицы
\usepackage{multirow} % Слияние строк в таблице

%%% Теоремы
\theoremstyle{plain} % Это стиль по умолчанию, его можно не переопределять.
\newtheorem{theoremm}{Теорема}[section]
\newtheorem{propositionn}[theoremm]{Утверждение}
\newtheorem{algorithmm}[theoremm]{Алгоритм}
\newenvironment{theorem}[1][Теорема]{%
    \begin{theoremm}[#1]$ $\par\nobreak\ignorespaces
    }{%
    \end{theoremm}
}
\newenvironment{proposition}[1][Утверждение]{%
    \begin{propositionn}[#1]$ $\par\nobreak\ignorespaces
    }{%
    \end{propositionn}
}
\newenvironment{algorithm}[1][Алгоритм]{%
    \begin{algorithmm}[#1]$ $\par\nobreak\ignorespaces
    }{%
    \end{algorithmm}
}

\theoremstyle{definition} % "Определение"
\newtheorem{corollary}{Следствие}[theoremm]
\newtheorem{problem}{Задача}[section]

\theoremstyle{remark} % "Примечание"
\newtheorem*{nonum}{Решение}

%%% Программирование
\usepackage{etoolbox} % логические операторы
\usepackage{listings} 

%%% Страница
\usepackage{extsizes} % Возможность сделать 14-й шрифт
\usepackage{geometry} % Простой способ задавать поля
	\geometry{top=25mm}
	\geometry{bottom=35mm}
	\geometry{left=35mm}
	\geometry{right=20mm}
 %
%\usepackage{fancyhdr} % Колонтитулы
% 	\pagestyle{fancy}
 	%\renewcommand{\headrulewidth}{0pt}  % Толщина линейки, отчеркивающей верхний колонтитул
% 	\lfoot{Нижний левый}
% 	\rfoot{Нижний правый}
% 	\rhead{Верхний правый}
% 	\chead{Верхний в центре}
% 	\lhead{Верхний левый}
%	\cfoot{Нижний в центре} % По умолчанию здесь номер страницы

\usepackage{setspace} % Интерлиньяж
%\onehalfspacing % Интерлиньяж 1.5
%\doublespacing % Интерлиньяж 2
%\singlespacing % Интерлиньяж 1

\usepackage{lastpage} % Узнать, сколько всего страниц в документе.

\usepackage{soul} % Модификаторы начертания

\usepackage{hyperref}
\usepackage[usenames,dvipsnames,svgnames,table,rgb]{xcolor}
\hypersetup{				% Гиперссылки
    unicode=true,           % русские буквы в раздела PDF
    pdftitle={Заголовок},   % Заголовок
    pdfauthor={Автор},      % Автор
    pdfsubject={Тема},      % Тема
    pdfcreator={Создатель}, % Создатель
    pdfproducer={Производитель}, % Производитель
    pdfkeywords={keyword1} {key2} {key3}, % Ключевые слова
    colorlinks=true,       	% false: ссылки в рамках; true: цветные ссылки
    linkcolor=red,          % внутренние ссылки
    citecolor=black,        % на библиографию
    filecolor=magenta,      % на файлы
    urlcolor=cyan           % на URL
}

\usepackage{csquotes} % Еще инструменты для ссылок

%\usepackage[style=authoryear,maxcitenames=2,backend=biber,sorting=nty]{biblatex}

\usepackage{multicol} % Несколько колонок

\usepackage{tikz} % Работа с графикой
\usepackage{pgfplots}
\usepackage{pgfplotstable}



\begin{document}

%\centering{\textbf{Образ продукта}}
%\vspace{5ex}
%
%\begin{tabular}{|p{0.25\linewidth}|m{0.25\linewidth}|m{0.25\linewidth}|m{0.25\linewidth}|}
%    \hline
%    \multicolumn{3}{|l|}{\textbf{Проект: }} & \textbf{Дата составления: } \today \\
%    
%    \hline
%    \multicolumn{4}{|p{\linewidth}|}{\textbf{Продукт проекта: }
%    \input{product.tex}} \\
%
%    \hline
%    \multicolumn{4}{|l|}{\textbf{Цель проекта:}} \\
%    
%    \hline
%    \textbf{Целевая аудитория} & 
%    \textbf{Польза} 
%    \begin{turn}{0}\parbox{2cm}{\input{product.tex}n}\end{turn} & 
%    \textbf{Характеристика} & 
%    \textbf{Аналоги} \\
%    
%    \hline
%    \multicolumn{2}{|l|}{\textbf{Срок проекта:}} & \multicolumn{2}{l|}{\textbf{Предполагаемый бюджет проекта:}} \\
%    \hline
%\end{tabular}


\section*{Образ продукта} \hfill \textbf{Дата составления:} \today

\vspace{5ex}

\noindent
\textbf{Проект: } <название проекта>

\noindent
\textbf{Продукт проекта: } Сетевая кроссплатформенная игра --- настольный теннис.

\noindent
\textbf{Срок проекта: } 2 месяца.

\noindent
\textbf{Предполагаемый бюджет проекта: } 500 руб.


\subsection*{Цель проекта}

\noindent
Ускорение реабилитации больных путём проведения занятий в игровой форме.


\subsection*{Целевая аудитория}

\noindent
Люди, у которых из-за болезни ограничена способность к движению.


\subsection*{Польза}

\noindent
Проведение реабилитационных процедур в игровой форме может значительно увеличить их эффективность по сравнению с обычной гимнастикой. Также отображение воображаемых движений на экране может сильно помочь мозгу в восстановлении двигательных функций.


\subsection*{Характеристика}

\noindent
Спокойная игра в настольный теннис способствует расслаблению и эффективному отдыху, что позволит больным проще относиться к тяжёлым для них упражнениям в игре и даст стимул заниматься больше.

\noindent
Настольный теннис даёт большой простор для всевозможных упражнений. Игра может быть сколь угодно упрощена в самых разных аспектах.

\begin{itemize}
    \item Скорость полёта шарика и ход игрового времени могут быть настроены.
    
    \item Когда мячик подлетает к половине стола игрока, игровое время может замедляться, чтобы дать возможность игроку сделать движение, и классификатору --- чтобы его распознать.
    
    \item  В простейшем варианте игра может сама "играть" в себя. Человеку нужно будет только сделать какое-то движение (например, сжать кулак), чтобы бот, который играет за человека, сам отбил шарик.
    
    \item Игровой экран может быть поделён на зоны, поверхность стола --- также, чтобы вводить новые ситуации в игру, на которые нужно будет отвечать новыми движениями.
    
    \item В реальности техника игры очень комплексна (двигаются руки, кисти, даже пальцы участвуют в игре), что даёт большой простор на введение новых движений.
    
    \item Игрок будет видеть, как ракетка на экране отвечает его движениям, что важно для способствования восстановлению двигательных функций.
    
    \item Возможность сетевой игры вводит элемент конкуренции, что способствует большей целеустремлённости к занятиям и выздоровлению.
\end{itemize}


\subsection*{Аналоги}

\noindent
Существует много компьютерных эмуляций спортивных игр, в том числе и тенниса. Но все реализации не рассчитаны на кастомизацию игрового процесса. Также они не приспособлены для работы с нейроинтерфейсом, так как ему нужно время на распознавание движений, а все игры требуют мгновенной реакции игрока.

\end{document} % конец документа











