%\documentclass[handout]{beamer} % Раздаточный материал (на слайдах всё сразу)
%\documentclass[aspectratio=169]{beamer} % Соотношение сторон

%\usetheme{Berkeley} % Тема оформления
%\usetheme{Bergen}
%\usetheme{Szeged}

%\usecolortheme{beaver} % Цветовая схема
%\useinnertheme{circles}
%\useinnertheme{rectangles}

%%% Работа с русским языком
\usepackage{cmap}					% поиск в PDF
\usepackage{mathtext} 				% русские буквы в формулах
\usepackage[T2A]{fontenc}			% кодировка
\usepackage[utf8]{inputenc}			% кодировка исходного текста
\usepackage[english,russian]{babel}	% локализация и переносы

%% Beamer по-русски
\newtheorem{rtheorem}{Теорема}
\newtheorem{rproof}{Доказательство}
\newtheorem{rexample}{Пример}

%%% Дополнительная работа с математикой
\usepackage{amsmath,amsfonts,amssymb,amsthm,mathtools} % AMS
\usepackage{icomma} % "Умная" запятая: $0,2$ --- число, $0, 2$ --- перечисление

%% Номера формул
\mathtoolsset{showonlyrefs=true} % Показывать номера только у тех формул, на которые есть \eqref{} в тексте.
%\usepackage{leqno} % Нумерация формул слева

%% Свои команды
\DeclareMathOperator{\sgn}{\mathop{sgn}}

%% Перенос знаков в формулах (по Львовскому)
\newcommand*{\hm}[1]{#1\nobreak\discretionary{}
{\hbox{$\mathsurround=0pt #1$}}{}}

%%% Работа с картинками
\usepackage{graphicx}  % Для вставки рисунков
\graphicspath{{images/}{images2/}}  % папки с картинками
\setlength\fboxsep{3pt} % Отступ рамки \fbox{} от рисунка
\setlength\fboxrule{1pt} % Толщина линий рамки \fbox{}
\usepackage{wrapfig} % Обтекание рисунков текстом

%%% Работа с таблицами
\usepackage{array,tabularx,tabulary,booktabs} % Дополнительная работа с таблицами
\usepackage{longtable}  % Длинные таблицы
\usepackage{multirow} % Слияние строк в таблице

%%% Программирование
\usepackage{etoolbox} % логические операторы

%%% Другие пакеты
\usepackage{lastpage} % Узнать, сколько всего страниц в документе.
\usepackage{soul} % Модификаторы начертания
\usepackage{csquotes} % Еще инструменты для ссылок
%\usepackage[style=authoryear,maxcitenames=2,backend=biber,sorting=nty]{biblatex}
\usepackage{multicol} % Несколько колонок

%%% Картинки
\usepackage{tikz} % Работа с графикой
\usepackage{pgfplots}
\usepackage{pgfplotstable}